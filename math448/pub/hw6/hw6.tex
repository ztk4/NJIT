
% This LaTeX was auto-generated from MATLAB code.
% To make changes, update the MATLAB code and republish this document.

\documentclass{article}
\usepackage{graphicx}
\usepackage{color}

\sloppy
\definecolor{lightgray}{gray}{0.5}
\setlength{\parindent}{0pt}

\begin{document}

    
    
\section*{Zachary Kaplan}

\begin{par}
Math 448 12/6/18 HW 6
\end{par} \vspace{1em}

\subsection*{Contents}

\begin{itemize}
\setlength{\itemsep}{-1ex}
   \item Setup
   \item 9.12
   \item 9.18
\end{itemize}


\subsection*{Setup}

\begin{par}
Seed the random generator so that consecutive runs produce the same results.
\end{par} \vspace{1em}
\begin{verbatim}
rng(448120618);
\end{verbatim}


\subsection*{9.12}

\begin{verbatim}
% Common variables
n = 100;            % Sample size.
U = rand(1, n);     % Uniforms.
X = exp(U.^2);  % X = h(U) is our raw estimator.
Y = exp(U);     % Y = e^U is our control variate.

% Part b).
% Precalculated means.
Xbar = mean(X);
Ybar = mean(Y);
% First estimate c^*.
c_hat = - sum((X - Xbar) .* (Y - Ybar)) / sum((Y - Ybar).^2);

% Precalculate sample covariance.
cov_hat = sum((X - Xbar) .* (Y - Ybar)) / (n-1);
% Now estimate var of control variate estimator.
% NOTE: var normalizes by N-1 and is therefore a sample estimators.
control_var_hat = 1/n * (var(X) - cov_hat^2/var(Y));

fprintf('Control Variate: c^* ~= %f, var ~= %f\n', c_hat, control_var_hat);

% Part c).
% We just need to find the sample variance of n evaluations of the
% anithetic estimator.
antithetic_var_hat = var(exp(U.^2) .* (1 + exp(1 - 2*U)) / 2);

fprintf('Antithetic Variate: var ~= %f\n', antithetic_var_hat);
\end{verbatim}

        \color{lightgray} \begin{verbatim}Control Variate: c^* ~= -0.927313, var ~= 0.000134
Antithetic Variate: var ~= 0.029007
\end{verbatim} \color{black}
    

\subsection*{9.18}

\begin{verbatim}
% Common variables.
n = 1e2;

% e) - (RAW).

U1 = rand(1, n);
U2 = rand(1, n);

theta_raw = (2*norminv(U1) + norminv(U2) + 1) > 1;
fprintf('E[theta_raw] ~= %f, Var[theta_raw] ~= %f\n', ...
        mean(theta_raw), var(theta_raw));

% f) - (C)onditional.

U = rand(1, n);

theta_c = 1 - normcdf(norminv(U)/2);
fprintf('E[theta_c] ~= %f, Var[theta_c] ~= %f\n', ...
        mean(theta_c), var(theta_c));

% g) - (A)ntithetic (C)onditional (skipped b/c constant).

% h) - (C)ontrol (C)onditional.

U = rand(1, n);
X = 1 - normcdf(norminv(U)/2);  % The conditional estimator.
Y = norminv(U);                 % The control variate.

Xbar = mean(X);
Ybar = mean(Y);
c_hat = - sum((X - Xbar) .* (Y - Ybar)) / sum((Y - Ybar).^2);

% Actually evaluate the estimator n times to find sample mean and var.
theta_cc = X + c_hat*Y;

fprintf('E[theta_cc] ~= %f, Var[theta_cc] ~= %f\n', ...
        mean(theta_cc), var(theta_cc));
\end{verbatim}

        \color{lightgray} \begin{verbatim}E[theta_raw] ~= 0.520000, Var[theta_raw] ~= 0.252121
E[theta_c] ~= 0.497022, Var[theta_c] ~= 0.033256
E[theta_cc] ~= 0.499283, Var[theta_cc] ~= 0.000164
\end{verbatim} \color{black}
    


\end{document}
    
