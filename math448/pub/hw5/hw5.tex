
% This LaTeX was auto-generated from MATLAB code.
% To make changes, update the MATLAB code and republish this document.

\documentclass{article}
\usepackage{graphicx}
\usepackage{color}

\sloppy
\definecolor{lightgray}{gray}{0.5}
\setlength{\parindent}{0pt}

\begin{document}

    
    
\section*{Zachary Kaplan}

\begin{par}
Math 448 11/29/18 HW 5
\end{par} \vspace{1em}

\subsection*{Contents}

\begin{itemize}
\setlength{\itemsep}{-1ex}
   \item Setup
   \item 8.4
   \item 8.5
   \item 8.15
\end{itemize}


\subsection*{Setup}

\begin{par}
Seed the random generator so that consecutive runs produce the same results.
\end{par} \vspace{1em}
\begin{verbatim}
rng(448112918);
\end{verbatim}


\subsection*{8.4}

\begin{verbatim}
runs = 1e5;
N = zeros(1, runs);
bound = 0.1;
for i = 1:runs
    % Generate normals [the below is equiv to norminv(rand(1, n))].
    n = 100;               % Current sample size.
    Z = randn(1, n);       % Normals.
    sum_z = sum(Z);        % Current sum.
    sum_sq_z = sum(Z.^2);  % Current sum of squares.

    % The below condition is a rearrangement of the negation of
    % S_n/sqrt(n) < 0.1
    while sum_sq_z - (sum_z^2)/n >= bound^2 * n*(n-1)
        z = randn(1, 1);   % Generate a new normal.
        % Accumulate.
        sum_z = sum_z + z;
        sum_sq_z = sum_sq_z + z^2;
        % Increment sample size.
        n = n + 1;
    end

    N(i) = n;
end

% Not really required but I was curious.
% Constructs a CI about our estimate by dividing the values into 40 samples
% and using the variance of their means.
% This is a two-sided 95% CI.
batches = 40;
bsize = runs/batches;
Nb = zeros(1, batches);
for b = 1:batches
    Nb(b) = mean(N((b-1)*bsize + 1:b*bsize));
end
fprintf('CI(95%%) for n: %f +/- %f\n', ...
        mean(N), tinv(1-(1-.95)/2, batches - 1)*std(Nb)/sqrt(batches));

% Regular Output
fprintf('Sample Mean of n = %f, Sample Variance of n = %f\n', ...
        mean(N), var(N));
fprintf('The last trial had n = %d, mu = %f, and sigma^2 = %f\n', ...
        n, sum_z/n, (sum_sq_z - sum_z^2/n)/(n-1));
\end{verbatim}

        \color{lightgray} \begin{verbatim}CI(95%) for n: 105.211350 +/- 0.055003
Sample Mean of n = 105.211350, Sample Variance of n = 60.776529
The last trial had n = 104, mu = -0.200370, and sigma^2 = 1.031204
\end{verbatim} \color{black}
    

\subsection*{8.5}

\begin{verbatim}
runs = 1e3;
N = zeros(1, runs);
bound = 0.01;
for i = 1:runs
    % Generate normals [the below is equiv to norminv(rand(1, n))].
    n = 100;               % Current sample size.
    Z = randn(1, n);       % Normals.
    sum_z = sum(Z);        % Current sum.
    sum_sq_z = sum(Z.^2);  % Current sum of squares.

    % The below condition is a rearrangement of the negation of
    % S_n/sqrt(n) < 0.01
    while sum_sq_z - (sum_z^2)/n >= bound^2 * n*(n-1)
        z = randn(1, 1);   % Generate a new normal.
        % Accumulate.
        sum_z = sum_z + z;
        sum_sq_z = sum_sq_z + z^2;
        % Increment sample size.
        n = n + 1;
    end

    N(i) = n;
end

% Not really required but I was curious.
% Constructs a CI about our estimate by dividing the values into 40 samples
% and using the variance of their means.
% This is a two-sided 95% CI.
batches = 40;
bsize = runs/batches;
Nb = zeros(1, batches);
for b = 1:batches
    Nb(b) = mean(N((b-1)*bsize + 1:b*bsize));
end
fprintf('CI(95%%) for n: %f +/- %f\n', ...
        mean(N), tinv(1-(1-.95)/2, batches - 1)*std(Nb)/sqrt(batches));

% Regular Output.
fprintf('Sample Mean of n = %f, Sample Variance of n = %f\n', ...
        mean(N), var(N));
fprintf('The last trial had n = %d, mu = %f, and sigma^2 = %f\n', ...
        n, sum_z/n, (sum_sq_z - sum_z^2/n)/(n-1));
\end{verbatim}

        \color{lightgray} \begin{verbatim}CI(95%) for n: 9998.593000 +/- 8.476926
Sample Mean of n = 9998.593000, Sample Variance of n = 20095.785136
The last trial had n = 9927, mu = 0.006907, and sigma^2 = 0.992577
\end{verbatim} \color{black}
    

\subsection*{8.15}

\begin{verbatim}
% Data values
X = [5, 4, 9, 6, 21, 17, 11, 20, 7, 10, 21, 15, 13, 16, 8];
n = length(X);      % Size of vector.
R = 1e5;            % Number of evaluations of g.
g = @(X) var(X);    % The function that computes our estimator on a sample.
% Vectorized inverse of the emperical CDF Fe.
% Takes a vectors of uniforms and returns a vectors of samples ~ Fe.
Fe_inv = @(U) X(ceil(n*U));  % Recall matlab is 1-indexed.
% The true variance of Fe_inv.
theta_fe = var(X, 1);  % NOTE: the 1 normalizes by N (instead of N-1).

mse_ests = zeros(1, R);
for r = 1:R
    mse_ests(r) = (g(Fe_inv(rand(1, n))) - theta_fe)^2;
end

fprintf('MSE_Fe Estimate: %f\n', mean(mse_ests));
\end{verbatim}

        \color{lightgray} \begin{verbatim}MSE_Fe Estimate: 58.281434
\end{verbatim} \color{black}
    


\end{document}
    
