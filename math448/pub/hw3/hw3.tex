
% This LaTeX was auto-generated from MATLAB code.
% To make changes, update the MATLAB code and republish this document.

\documentclass{article}
\usepackage{graphicx}
\usepackage{color}

\sloppy
\definecolor{lightgray}{gray}{0.5}
\setlength{\parindent}{0pt}

\begin{document}

    
    
\section*{Zachary Kaplan}

\begin{par}
Math 448 10/18/18 HW 3
\end{par} \vspace{1em}

\subsection*{Contents}

\begin{itemize}
\setlength{\itemsep}{-1ex}
   \item 4.17
\end{itemize}


\subsection*{4.17}

\begin{verbatim}
N = 10;  % The number of samples of X we desire.

% First let's calculate the first N values of the text's random sequence.
x = zeros(1, N);
x(1:2) = [ 23 66 ];  % x_1 and x_2 are hard-coded.
for n = 3:N
    x(n) = mod(3*x(n-1) + 5*x(n-2), 100);
end

% Then let's generate samples of X.
X = zeros(1, N);
for n = 1:N
    j = 1;
    f1 = 1/4;
    f2 = 4/9;
    p_cum = f1 + 3/8 * f2;
    U = x(n)/100;

    while U > p_cum
        f1 = 1/2 * f1;
        f2 = 2/3 * f2;
        p_cum = p_cum + f1 + 3/8 * f2;
        j = j + 1;
    end

    X(n) = j;
end

% Display our samples of X.
X
\end{verbatim}

        \color{lightgray} \begin{verbatim}
X =

     1     3     1     3     3     2     2     1     1     1

\end{verbatim} \color{black}
    


\end{document}
    
