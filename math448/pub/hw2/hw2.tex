
% This LaTeX was auto-generated from MATLAB code.
% To make changes, update the MATLAB code and republish this document.

\documentclass{article}
\usepackage{graphicx}
\usepackage{color}

\sloppy
\definecolor{lightgray}{gray}{0.5}
\setlength{\parindent}{0pt}

\begin{document}

    
    
\section*{Zachary Kaplan}

\begin{par}
Math 448 10/4/18 HW 2
\end{par} \vspace{1em}

\subsection*{Contents}

\begin{itemize}
\setlength{\itemsep}{-1ex}
   \item 3.2
   \item 3.3
   \item 3.6
   \item 3.11
   \item 3.13
   \item Helper Functions
\end{itemize}


\subsection*{3.2}

\begin{verbatim}
xn = zeros(1, 11);
xn(1) = 3;  % NOTE: matlab uses 1-indexing, so xn(i) = x_{i-1}.
for i = 1:10
    xn(i + 1) = mod(5*xn(i) + 7, 200);
end

% Display x_1 through x_10.
xn(2:end)
\end{verbatim}

        \color{lightgray} \begin{verbatim}
ans =

    22   117   192   167    42    17    92    67   142   117

\end{verbatim} \color{black}
    

\subsection*{3.3}

\begin{verbatim}
t = 6.31656;
N = 1e6;
U = rand(1, N);
G = exp(exp(U));

theta_N = mean(G)
err = abs(theta_N - t)
\end{verbatim}

        \color{lightgray} \begin{verbatim}
theta_N =

    6.3132


err =

    0.0034

\end{verbatim} \color{black}
    

\subsection*{3.6}

\begin{verbatim}
t = 1/2;
N = 1e6;
U = rand(1, N);
g = @(x) x ./ ((1 + x.^2).^2);
G = g(1./U - 1);

theta_N = mean(G./(U.^2))
err = abs(theta_N - t)
\end{verbatim}

        \color{lightgray} \begin{verbatim}
theta_N =

    0.5001


err =

   1.4247e-04

\end{verbatim} \color{black}
    

\subsection*{3.11}

\begin{verbatim}
N = 1e6;
U = rand(1, N);
Xa = U;
Xb = U.^2;
Y = sqrt(1 - U.^2);
mu_xa = mean(Xa);
mu_xb = mean(Xb);
mu_y = mean(Y);
sigma_xa = std(Xa);
sigma_xb = std(Xb);
sigma_y = std(Y);

theta_Na = mean((Xa - mu_xa) .* (Y - mu_y)) / (sigma_xa * sigma_y)
theta_Nb = mean((Xb - mu_xb) .* (Y - mu_y)) / (sigma_xb * sigma_y)
\end{verbatim}

        \color{lightgray} \begin{verbatim}
theta_Na =

   -0.9214


theta_Nb =

   -0.9836

\end{verbatim} \color{black}
    

\subsection*{3.13}

\begin{verbatim}
M = 1e6;
N = arrayfun(@(~) sample_N, 1:M);

theta_M = mean(N)
for i = [0 1 2 3 4 5 6]
    fprintf('theta^%d_M = \n\n    %f\n\n', i, mean(N == i))
end
\end{verbatim}


\subsection*{Helper Functions}

\begin{verbatim}
function N = sample_N()
    cumm_prod = 1;
    i = 0;
    while cumm_prod >= exp(-3)
        cumm_prod = cumm_prod * rand;
        i = i + 1;
    end

    N = i - 1;
end
\end{verbatim}

        \color{lightgray} \begin{verbatim}
theta_M =

    2.9968

theta^0_M = 

    0.049968

theta^1_M = 

    0.150014

theta^2_M = 

    0.224397

theta^3_M = 

    0.223496

theta^4_M = 

    0.167918

theta^5_M = 

    0.100429

theta^6_M = 

    0.050216

\end{verbatim} \color{black}
    


\end{document}
    
