
% This LaTeX was auto-generated from MATLAB code.
% To make changes, update the MATLAB code and republish this document.

\documentclass{article}
\usepackage{graphicx}
\usepackage{color}

\sloppy
\definecolor{lightgray}{gray}{0.5}
\setlength{\parindent}{0pt}

\begin{document}

    
    
\subsection*{Contents}

\begin{itemize}
\setlength{\itemsep}{-1ex}
   \item Zachary Kaplan
   \item Parabolic PDE Instance
   \item C.4.1.1 - Dimensionless PDE
\end{itemize}


\subsection*{Zachary Kaplan}

\begin{par}
Math 440 Computational Lab \#3 4/11/19
\end{par} \vspace{1em}
\begin{verbatim}
%
% An online version of this file can be found published to:
% https://www.overleaf.com/read/zgmxqsryykhj
%
% *NOTE*: To render this file locally, please prepend the following package
%         dependencies to the latex generated by matlab publish:
%
%  \usepackage[margin=1in]{geometry}
%  \usepackage{amsmath}
%  \usepackage{amssymb}
%  \usepackage{microtype}
%  \usepackage{csquotes}
%  \usepackage{bm}
%
\end{verbatim}


\subsection*{Parabolic PDE Instance}

\begin{par}
In this lab, we will consider the PDE describing the diffusion of heat through a metallic rod.
\end{par} \vspace{1em}
\begin{par}
Let $L$ [ $m$ ] denote the length of the rod, $T$ [ $C$ ] denote the temperature of the rod across space and time, $\rho$ [ $kg/m^3$ ] denote the density of the rod, $C_p$ [ $J/(kg \cdot C)$ ] denote the heat capacity of the rod, and $k$ [ $J/(m \cdot s \cdot C)$ ] denote the thermal conductivity of the rod. We will consider this physical system over the 1-D spatial domain $x \,[m] \in \Omega = (0, L)$, and the temporal domain $t\,[s] \in (0, \infty)$. The following PDE then describes the difussion of heat through the rod:
\end{par} \vspace{1em}
\begin{par}
$$
  \rho C_p \frac{\partial T}{\partial t} =
      k \frac{\partial^2 T}{\partial x^2},
      \quad \forall (x, t) \in \Omega \times (0, \infty).
$$
\end{par} \vspace{1em}
\begin{par}
We want to consider this rod where a temperature pulse of amplitude $T_0$ occurs at the left endpoint $x = 0$ for time $t \in (0, t_p)$. On the right endpoint, we assuming the rod is isolated and so heat does not flow through where $x = L$. As such, our boundary and initial conditions are:
\end{par} \vspace{1em}
\begin{par}
$$ \begin{gathered}
  T(0, t) =
    \begin{cases} T_0 & 0 \le t \le t_p \\ 0 & t > t_p \end{cases}, \quad
  \frac{\partial T}{\partial x}(L, t) = 0, \\
  T(x, 0) = \begin{cases} T_0 & x = 0 \\ 0 & 0 < x \le L \end{cases}.
\end{gathered} $$
\end{par} \vspace{1em}


\subsection*{C.4.1.1 - Dimensionless PDE}

\begin{par}
In this section, we show that with the proper rescaling of our domain and function $T$, we can represent the above constraints on $T$ with dimensionless equations. Consider dimensionless parameters $\xi$ for length and $\tau$ for time, and the dimensionless function $u$ where
\end{par} \vspace{1em}
\begin{par}
$$ \begin{aligned}x = L\xi, && t = t_p\tau, && T = T_0u\end{aligned}. $$
\end{par} \vspace{1em}
\begin{par}
If we look at the endpoints of our original domain we see that $\xi|_{x = 0} = 0, \xi|_{x = L} = 1,   \tau|_{t = 0} = 0, \tau|_{t \to \infty} \to \infty$. As such, our dimensionless spatial domain becomes $\xi \in \hat\Omega = (0, 1)$, and our dimensionaless temporal domain becomes $\tau \in (0, \infty)$. We can then write the partials of $T$ in terms of $u$, $\xi$, and $\tau$ as
\end{par} \vspace{1em}
\begin{par}
$$ \begin{aligned}
  \frac{\partial T}{\partial t} &=
    \frac{\partial T_0u}{\partial \tau}\frac{\partial \tau}{\partial t} =
    \frac{T_0}{t_p} \frac{\partial u}{\partial \tau}, \\
  \frac{\partial T}{\partial x} &=
    \frac{\partial T_0u}{\partial \xi^2} \frac{\partial \xi}{\partial x} =
  \frac{T_0}{L} \frac{\partial u}{\partial \xi}, \\
  \frac{\partial^2 T}{\partial x^2} &=
    \frac{\partial^2 T_0u}{\partial \xi^2}
       \left(\frac{\partial \xi}{\partial x}\right)^2 =
  \frac{T_0}{L^2} \frac{\partial^2 u}{\partial \xi^2}.
\end{aligned} $$
\end{par} \vspace{1em}
\begin{par}
Substituting these values into the original PDE, we find
\end{par} \vspace{1em}
\begin{par}
$$ \begin{gathered}
  \rho C_p \frac{T_0}{t_p} \frac{\partial u}{\partial \tau} =
    k \frac{T_0}{L^2} \frac{\partial^2 u}{\partial \xi^2} \\
  \implies \frac{\partial u}{\partial \tau} =
    a \frac{\partial^2 u}{\partial \xi^2},
  \quad (\xi, \tau) \in \hat\Omega \times (0, \infty),
\end{gathered} $$
\end{par} \vspace{1em}
\begin{par}
where
\end{par} \vspace{1em}
\begin{par}
$$ \begin{gathered}
 a = \frac{k t_p}{\rho C_p L^2} ~\text{has units}~
 \left[ \frac{J/(m \cdot s \cdot C) \cdot s}
        {kg/m^3 \cdot J/(kg \cdot C) \cdot m^2} \right] \equiv
 \left[ \frac{J}{J} \frac{1/m}{1/m^3 \cdot m^2} (1/s \cdot s)
   \frac{1/C}{1/C} \frac{1}{kg \cdot 1/kg} \right] \equiv [1]. \\
 \therefore a~\text{is dimensionless}.
\end{gathered} $$
\end{par} \vspace{1em}
\begin{par}
We can also substitute these equations into the boundary and initial conditions to obtain
\end{par} \vspace{1em}
\begin{par}
$$ \begin{aligned}
  T(0 = x(0), t(\tau)) &=
  \begin{cases}
    T_0 & 0 \le t_p \tau \le t_p \\ 0 & t_p \tau > t_p
  \end{cases} = T_0 u(0, \tau)
  &&\implies u(0, \tau) =
  \begin{cases}
    1 & 0 \le \tau \le 1 \\ 0 & \tau > 1
  \end{cases}, \\
  \frac{\partial T}{\partial x}(L = x(1), t(\tau)) &= 0 =
    \frac{T_0}{L} \frac{\partial u}{\partial \xi}(1, \tau)
  &&\implies \frac{\partial u}{\partial \xi}(1, \tau) = 0, \\
  T(x(\xi), 0 = t(0)) &=
    \begin{cases} T_0 & L\xi = 0 \\ 0 & 0 < L\xi \le L \end{cases} =
  T_0 u(\xi, 0)
  &&\implies u(\xi, 0) =
    \begin{cases} 1 & \xi = 0 \\ 0 & 0 < \xi \le 1
  \end{cases}.
\end{aligned} $$
\end{par} \vspace{1em}
\begin{par}
From this point forward, this lab will only consider this dimensionless problem, fixing the dimensionless parameter $a = 1$.
\end{par} \vspace{1em}



\end{document}
    
