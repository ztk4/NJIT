
% This LaTeX was auto-generated from MATLAB code.
% To make changes, update the MATLAB code and republish this document.

\documentclass{article}
\usepackage{graphicx}
\usepackage{color}

\sloppy
\definecolor{lightgray}{gray}{0.5}
\setlength{\parindent}{0pt}

\begin{document}

    
    
\section*{Zachary Kaplan}

\begin{par}
Math 440 Computational Lab \#4 5/3/19
\end{par} \vspace{1em}

\subsection*{Contents}

\begin{itemize}
\setlength{\itemsep}{-1ex}
   \item An Elliptic PDE Instance
   \item C.5.1.1 - Numerically Solve the PDE
\end{itemize}
\begin{par}
An online version of this file can be found published to: https://www.overleaf.com/read/wwdtnstrvpqz
\end{par} \vspace{1em}
\begin{par}
\textbf{NOTE}: To render this file locally, please prepend the following package         dependencies to the latex generated by matlab publish:
\end{par} \vspace{1em}

\begin{verbatim}\usepackage[margin=1in]{geometry}
\usepackage{amsmath}
\usepackage{amssymb}
\usepackage{microtype}
\usepackage{csquotes}
\usepackage{bm}\end{verbatim}
    \begin{par}
It is also recommended to use the following command to stop matrices from wrapping.
\end{par} \vspace{1em}

\begin{verbatim}\setcounter{MaxMatrixCols}{20}\end{verbatim}
    

\subsection*{An Elliptic PDE Instance}

\begin{par}
Consider a 2D metal block denoted by the region $\Omega = [0,4] \times [0, 2]$. The following PDE defines the temperature distribution throughout the block given von Neumann Boundary Conditions on the upper and lower boundaries, and Dirichlet Boundary Conditions on the left and right boundaries.
\end{par} \vspace{1em}
\begin{par}
$$ \begin{aligned}
 \nabla^2 T = 0, && \forall~(x,y) \in \Omega \\
 T(0, y) = 300, && T(4, y) = 600, && \forall~y \in [0, 2] \\
 \frac{\partial T}{\partial y}(x, 0) = 0,
   && \frac{\partial T}{\partial y}(x, 2) = 0,
   && \forall~x \in [0, 4] \\
\end{aligned} $$
\end{par} \vspace{1em}


\subsection*{C.5.1.1 - Numerically Solve the PDE}

\begin{par}
First, we discretize space using a uniform stepsize $h$ as suggested in the text, with ghost points along $y = -h$ and $y = 2+h$. This results in a grid of points $x_{ij}$ where $x_{ij} = (x_j, y_i) = (hj, h(i-1)),  \forall~ i = 0, 1, \ldots, M+1 = 2/h + 2~\text{and}~  j = 0, 1, \ldots, N+1 = 4/h$. We then define our approximate solution $T_{ij} = T(x_{ij}) = T(x_j, y_i)$.
\end{par} \vspace{1em}
\begin{par}
Now we define our FD approximations for the derivatives in our PDE system:
\end{par} \vspace{1em}
\begin{par}
$$ \begin{aligned}
 \nabla^2T(x_j, y_i) &\approx
   \frac{T_{i+1,j} - 2T_{ij} + T_{i-1,j}}{h} +
   \frac{T_{i,j+1} - 2T_{ij} + T_{i,j-1}}{h} \\
   &= \frac{1}{h}
   \left( T_{i+1,j} + T_{i-1,j} + T_{i,j+1} + T_{i,j-1} - 4T_{ij}\right)
   = 0 && \forall~i = 1, 2, \ldots, M~\text{and}~j= 1, 2, \ldots, N  \\
 \frac{\partial T}{\partial y}(x_j, 0)
   &\approx \frac{T_{2j} - T_{0j}}{2h} = 0
    && \forall~j = 1, 2, \ldots, N \\
 \frac{\partial T}{\partial y}(x_j, 2)
   &\approx \frac{T_{M+1,j} - T_{M-1,j}}{2h} = 0
    && \forall~j = 1, 2, \ldots, N \\
 T(0, y_i) &\approx T_{i0} = 300, \quad T(4, y_i) \approx T_{i,N+1} = 600
   && \forall~i = 1, 2, \ldots, M
\end{aligned} $$
\end{par} \vspace{1em}
\begin{par}
Notice that none of the above constraints operate on the values $T_{00}, T_{0,N+1}, T_{M+1,0}, T_{M+1,N+1}$. Since none of these values are in $\overline{\Omega}$, we just set them explicitly to 0.
\end{par} \vspace{1em}
\begin{par}
Let the vector $\bm{T} = \begin{pmatrix} T_{00}, T_{10}, \ldots,                           T_{M+1,0}, T_{01}, T_{11}, \ldots,                           T_{M,N+1}, T_{M+1,N+1} \end{pmatrix}^\top$ be a flattened representation of our approximate solution $T_{ij}$. Then we can represent the PDE constraints of our system as
\end{par} \vspace{1em}
\begin{par}
$$ \begin{aligned}
 A_1 \bm{T} &= \bm0_{NM}, \\
 A_1 &= \begin{pmatrix}
  0 &  1 & 0 & \bm{0}_{M-1} &
  1 & -4 & 1 & \bm{0}_{M-1} &
  0 &  1 & 0 & \multicolumn{5}{c}{\cdots} & 0 \\
  0 & 0 &  1 & 0 & \bm{0}_{M-1} &
      1 & -4 & 1 & \bm{0}_{M-1} &
      0 &  1 & 0 & \multicolumn{4}{c}{\cdots} & 0 \\
  0 & 0 &      0 & \ddots & \ddots & \ddots
        & \ddots & \ddots & \ddots & \ddots
        & \ddots & \ddots & 0 & \multicolumn{3}{c}{\cdots} & 0 \\
  \multicolumn{17}{c}{\vdots} \\
  \bm{0}_{M-1} & 0 &  1 & 0 & \bm{0}_{M-1} &
                 1 & -4 & 1 & \bm{0}_{M-1} &
                 0 &  1 & 0 & \multicolumn{4}{c}{\cdots} & 0 \\
  \bm{0}_{M-1} & 0 & 0 & 0 & 0 &  1 & 0 & \bm{0}_{M-1} &
                             1 & -4 & 1 & \bm{0}_{M-1} &
                             0 &  1 & 0 & \cdots & 0 \\
  \multicolumn{17}{c}{\vdots} \\
  0 & \multicolumn{4}{c}{\cdots} & 0 &  1 & 0 & \bm{0}_{M-1} &
                                   1 & -4 & 0 & \bm{0}_{M-1} &
                                   0 &  1 & 0 & 0 \\
  0 & \multicolumn{4}{c}{\cdots} & 0 & 0 &  1 & 0 & \bm{0}_{M-1} &
                                   1 & -4 & 0 & \bm{0}_{M-1} &
                                   0 &  1 & 0 \\
 \end{pmatrix}.
\end{aligned} $$
\end{par} \vspace{1em}
\begin{par}
Notice that $A_1 \in \mathbb{R}^{NM \times (N+2)(M+2)}$ and is made up of $N$ banded blocks of $M$ equations each. Each block shares the same band $\begin{pmatrix} 0 &  1 & 0 & \bm{0}_{M-1} &                       1 & -4 & 0 & \bm{0}_{M-1} &                       0 &  1 & 0 \end{pmatrix}$, and between each block there is an offset of 3 entries (as illustrated in the middle section of $A_1$ which shows the  border of block 1 and block 2). Another way to view this matrix is as a banded matrix with the above band justified at the top left corner where any row $k$ s.t. $k \equiv M+1 (\mod M+2)$ or $k \equiv M+2 (\mod M+2)$ is omitted.
\end{par} \vspace{1em}
\begin{par}
The von Neumann BCs can similarly be represented by a matrix system:
\end{par} \vspace{1em}
\begin{par}
$$ \begin{aligned}
 A_2 \bm{T} &= \bm{0}_{2N} \\
 A_2 &= \begin{pmatrix}
  \bm{0}_{M+2}            & -1 & 0 & 1 & 0 & \cdots & 0 \\
  \bm{0}_{2(M+2) - 3}     & -1 & 0 & 1 & 0 & \cdots & 0 \\
  \bm{0}_{2(M+2)}         & -1 & 0 & 1 & 0 & \cdots & 0 \\
  \multicolumn{7}{c}{\vdots} \\
  \bm{0}_{N(M+2)}         & -1 & 0 & 1 & 0 & \cdots & 0 \\
  \bm{0}_{(N+1)(M+2) - 3} & -1 & 0 & 1 & \multicolumn{3}{c}{\bm{0}_{M+2}}
 \end{pmatrix}.
\end{aligned} $$
\end{par} \vspace{1em}
\begin{par}
Note that $A_2 \in \mathbb{R}^{2N \times (N+2)(M+2)}$ and is essentially a matrix with band $\begin{pmatrix} -1 & 0 & 1 \end{pmatrix}$ justified at the top left corner where only rows $k$ s.t. $k \equiv 1 (\mod M+2)$ or $k \equiv M (\mod M+2)$ are included. Furthermore, rows $k = 1, M, (N+1)(M+2) + 1, (N+2)(M+2) - 2$ are omitted depite satisfying the modulus equations above (they would correspond to the BCs at x\_0 and x\_\{N+1\}).
\end{par} \vspace{1em}
\begin{par}
Finally, we can write our Dirichlet BCs and the constraints on the corners of our discrete grid as a simple matrix system:
\end{par} \vspace{1em}
\begin{par}
$$ \begin{aligned}
 A_3 \bm{T} &= \bm{b} \\
 A_3 &= \begin{pmatrix}
  0 & 1 & 0 & 0 & 0 & \cdots & 0 \\
  0 & 0 & 1 & 0 & 0 & \cdots & 0 \\
  0 & 0 & 0 & \ddots & 0 & \cdots & 0 \\
  \multicolumn{7}{c}{\vdots} \\
  \bm{0}_M & 1 & 0 & 0 & 0 & \cdots & 0 \\
  0 & \cdots & 0 & 0 & 1 & \multicolumn{2}{r}{\bm{0}_M} \\
  0 & \cdots & 0 & 0 & 0 & \ddots & \vdots \\
  \multicolumn{7}{c}{\vdots} \\
  0 & \cdots & 0 & 0 & 0 & 1 & 0 \\
  1 & 0 & 0 & 0 & 0 & \cdots & 0 \\
  \bm0_{M+1} & 1 & 0 & 0 & 0 & \cdots & 0 \\
  0 & \cdots & 0 & 0 & 0 & 1 & \bm0_{M+1} \\
  0 & \cdots & 0 & 0 & 0 & 0 & 1 \\
 \end{pmatrix},
 b = \begin{pmatrix}
   300 \cdot \bm{1}_{N}^\top \\ 600 \cdot \bm{1}_{N}^\top \\
   0 \\ 0 \\ 0 \\ 0
 \end{pmatrix}
\end{aligned} $$
\end{par} \vspace{1em}



\end{document}
    
